% Plantilla para problemas de OIA
% por Carolina González y Martín Villagra
% (hecho a ojo)

\documentclass[fontsize=13pt, paper=a4, DIV=calc]{scrartcl}
\usepackage{silence}
\WarningFilter{scrartcl}{Usage of package `fancyhdr'}
\WarningFilter{scrartcl}{Usage of package `titlesec'}
\WarningFilter{scrartcl}{Activating an ugly workaround for a missing}
\WarningFilter{titlesec}{Non standard sectioning command detected}
\usepackage[spanish]{babel}
\usepackage[utf8]{inputenc}
\usepackage[pdftex]{graphicx}
\usepackage{amsmath,amssymb,amsfonts,latexsym,cancel}
\usepackage{color}
\usepackage{rotating}
\usepackage{array}
\usepackage{multirow}
\usepackage{multicol}
\usepackage{pbox}
\usepackage{tikz}
\usepackage{makeidx}
\usepackage{eso-pic}
\usepackage{lastpage}
\usepackage{fancyhdr}
\usepackage{listings}
\usepackage{titlesec}
\usepackage{enumitem}
\usepackage{helvet}
\usepackage[T1]{fontenc}
\usepackage[helvet]{sfmath}
\everymath={\mathsf\mathbf}
\decimalpoint
\textheight = 26cm
\textwidth = 18cm
\topmargin = -1.5cm
\oddsidemargin = -1.5cm
\hoffset = 0.55cm
\fancyheadoffset{-1pt}
\pagestyle{fancy}
\lfoot{\scriptsize\version}
\cfoot{}
\rfoot{\scriptsize hoja {\thepage} de \pageref{LastPage}}
\renewcommand{\headrule}{\vspace*{-9pt}\rule{\textwidth}{0.2pt}\\[\dimexpr-\baselineskip+0.5pt]\rule{\textwidth}{0.2pt}}
\renewcommand{\footrulewidth}{0.2pt}
\setlength{\footskip}{20pt}
\AddToShipoutPicture{
	\begin{tikzpicture}
		\draw[white] (0,0) rectangle (0,0);
		\draw[very thick] (0.3,0.8) rectangle (19.45,28.9);
	\end{tikzpicture}
}
\renewcommand{\familydefault}{\sfdefault}
\setlength{\parindent}{0.5cm}
\setlength{\columnsep}{30pt}
\setlength{\columnseprule}{0.2pt}
\setlength{\headsep}{0.1in}
\titleformat{\section} {\normalfont}{}{0pt}{\textbf}
\titlespacing*{\section}{0pt}{10pt}{0pt}
\newcommand{\caja}[2]{
	{\fontfamily{pcr}\selectfont\large
	\hspace*{16pt}{\setlength{\fboxrule}{0pt}\fboxsep4pt\fbox{\pbox{\textwidth}{
		\hspace*{16pt}{\setlength{\fboxrule}{0.5pt}\fboxsep8pt\framebox[#1][l]{\pbox{\textwidth}{#2}}}
	}}}
	}\\\par
}
\renewcommand*{\maketitle}{
	\begin{center}
		\textbf{\nombre}\\
		\textit{\footnotesize Contribución de \author}
	\end{center}
}
\newlist{lista}{itemize}{2}
\setlist[lista]{label=$\bullet\,\,$,itemindent=12pt, leftmargin=*, itemsep=-7pt, topsep=-7pt}
\newlist{lista2}{itemize}{2}
\setlist[lista2]{label=$\circ$,itemindent=-8pt, leftmargin=30pt, itemsep=-7pt, topsep=-7pt}
\newlist{itemletras}{enumerate}{2}
\setlist[itemletras]{label=\alph*.$\;$, leftmargin=20pt, itemsep=-1pt, topsep=-7pt}
\setlength{\parskip}{5pt}
\hyphenpenalty=2000 \emergencystretch=20pt
\lhead{\textbf{ \niveldia{ } Problema \numero}}
\chead{\normalfont \ttfamily \archivo}
\rhead{\textbf {\contest}}
\newcommand*{\breakcolumn}{\null\vspace*{\fill}\columnbreak}

% Colores
\definecolor{gray}{rgb}{0.5, 0.5, 0.5}
\definecolor{pink}{rgb}{1, 0.25, 0.75}
\definecolor{dark green}{rgb}{0, 0.5, 0}
\definecolor{orange}{rgb}{1, 0.75, 0}
\definecolor{dark red}{rgb}{0.75, 0, 0}
\definecolor{purple}{rgb}{0.65, 0, 1}
\definecolor{dark blue}{rgb}{0, 0, 0.5}

% Espacios truchos
\def\ne{\vspace{-12pt}} % negative enter
\def\se{$ $\\}		    % simple enter
\def\de{\se\newline}	    % doble enter
\def\te{\de\newline}	    % triple enter
\def\tab{\hspace*{16pt}} % tab

% Separar en sílabas
\hyphenation{oia-no-va}

% Subrayado en color
% {\color{pink}\underline{\color{black}TEXTO}}\\

\begin{document}
%=============================MODIFICABLE=============================
%*********************************************************************
\def\contest{OIA Certamen Nacional}
\def\nombre{Curry o Carry}
\def\author{Lautaro Lasorsa}
\def\niveldia{}
\def\numero{}
\def\archivo{curry}% Nombre del archivo
\def\version{}
%*********************************************************************
%*********************************************************************


\maketitle
\begin{multicols*}{2}

\section{Descripción del problema}

Al escribir mensajes de texto es posible equivocarnos al escribir una palabra, por ejemplo poniendo una letra donde iba otra. Obviamente esto puede llevar a qué lo que estamos escribiendo cambie completamente de significado, por ejemplo es muy distinto decír "Me gusta el curry" que "Me gusta el carry".

Es por esto que, partiendo del diccionario de palabras existentes, se quiere saber el impacto que pueden tener estos errores. Como una primer aproximación se decidio representar el significado de cada palabra como un entero entre 1 y 1.000.000. Las palabras que no pertenecen al diccionario no son importantes porque no pueden llevar a confusiones.

Para medir el posible impacto de estos errores se buscará sumar la diferencia de significado (entre los enteros que representan el significado) para cada par de palabras que tienen la misma longitud y difieren en una sola letra.

\section{Detalles de implementación}

Debes implementar la función \texttt{curry(palabras,significados)} donde \texttt{palabras} es un {VECTOR DE CADENAS DE CARACTERES} y \texttt{singificados} es un {VECTOR DE ENTEROS}. Ambos son de longitud $N$, siendo $palabras[i]$ la i-esima palabra del diccionario y $significados[i]$ el número que representa su significado. Se garantiza que en el diccionario no hay 2 palabras iguales y qué cada palabra consta solo de letras mayusculas y minusculas del alfabeto ingles, aunque puede haber 2 palabras distintas con el mismo significado. 

La función debe retornar un \texttt{ENTERO LARGO} que sea el posible error total fruto del error que estamos estudiando.


\section{Evaluador local}
El evaluador local leera primero el entero $N$.
Luego leera $N$ pares de líneas, siendo en cada par la primer línea una palabra y en la siguiente el significado asociado a la misma.

Mostrará por pantalla el número que devuelva el llamado a \texttt{curry(palabras, significados)}

\section{Cotas}
  \begin{itemize}
     \item $ 1 \leq N \leq 100.000$
     \item Siendo $M$ la suma de las longitudes de las palabras del diccionario
     $1 \leq M \leq 300.000$
  \end{itemize}

\section{Ejemplos}
Si el evaluador local recibe la siguiente entrada:

\caja{5.7cm}{\small
4
curry\\
1000000\\
carry\\
1\\
curro\\
10000\\
carro\\
2
}
Entonces deberá devolver:

\caja{5.7cm}{\small
1999998
}
\begin{enumerate}
    \item $1 \leq N, max(|palabras[i]|) \leq 100$ (10 puntos)
    \item $1 \leq N , max(|palabras[i]|)\leq 500$ (20 puntos)
    \item Todas las palabras tienen la misma longitud (10 puntos)
    \item $1 \leq significado_i \leq 2$ (10 puntos)
    \item Sin restricciones adicionales (50 puntos)
\end{enumerate}

\end{multicols*}
\end{document}
 
