% Plantilla para problemas de OIA
% por Carolina González y Martín Villagra
% (hecho a ojo)

\documentclass[fontsize=13pt, paper=a4, DIV=calc]{scrartcl}
\usepackage{silence}
\WarningFilter{scrartcl}{Usage of package `fancyhdr'}
\WarningFilter{scrartcl}{Usage of package `titlesec'}
\WarningFilter{scrartcl}{Activating an ugly workaround for a missing}
\WarningFilter{titlesec}{Non standard sectioning command detected}
\usepackage[spanish]{babel}
\usepackage[utf8]{inputenc}
\usepackage[pdftex]{graphicx}
\usepackage{amsmath,amssymb,amsfonts,latexsym,cancel}
\usepackage{color}
\usepackage{rotating}
\usepackage{array}
\usepackage{multirow}
\usepackage{multicol}
\usepackage{pbox}
\usepackage{tikz}
\usepackage{makeidx}
\usepackage{eso-pic}
\usepackage{lastpage}
\usepackage{fancyhdr}
\usepackage{listings}
\usepackage{titlesec}
\usepackage{enumitem}
\usepackage{helvet}
\usepackage[T1]{fontenc}
\usepackage[helvet]{sfmath}
\everymath={\mathsf\mathbf}
\decimalpoint
\textheight = 26cm
\textwidth = 18cm
\topmargin = -1.5cm
\oddsidemargin = -1.5cm
\hoffset = 0.55cm
\fancyheadoffset{-1pt}
\pagestyle{fancy}
\lfoot{\scriptsize\version}
\cfoot{}
\rfoot{\scriptsize hoja {\thepage} de \pageref{LastPage}}
\renewcommand{\headrule}{\vspace*{-9pt}\rule{\textwidth}{0.2pt}\\[\dimexpr-\baselineskip+0.5pt]\rule{\textwidth}{0.2pt}}
\renewcommand{\footrulewidth}{0.2pt}
\setlength{\footskip}{20pt}
\AddToShipoutPicture{
	\begin{tikzpicture}
		\draw[white] (0,0) rectangle (0,0);
		\draw[very thick] (0.3,0.8) rectangle (19.45,28.9);
	\end{tikzpicture}
}
\renewcommand{\familydefault}{\sfdefault}
\setlength{\parindent}{0.5cm}
\setlength{\columnsep}{30pt}
\setlength{\columnseprule}{0.2pt}
\setlength{\headsep}{0.1in}
\titleformat{\section} {\normalfont}{}{0pt}{\textbf}
\titlespacing*{\section}{0pt}{10pt}{0pt}
\newcommand{\caja}[2]{
	{\fontfamily{pcr}\selectfont\large
	\hspace*{16pt}{\setlength{\fboxrule}{0pt}\fboxsep4pt\fbox{\pbox{\textwidth}{
		\hspace*{16pt}{\setlength{\fboxrule}{0.5pt}\fboxsep8pt\framebox[#1][l]{\pbox{\textwidth}{#2}}}
	}}}
	}\\\par
}
\renewcommand*{\maketitle}{
	\begin{center}
		\textbf{\nombre}\\
		\textit{\footnotesize Contribución de \author}
	\end{center}
}
\newlist{lista}{itemize}{2}
\setlist[lista]{label=$\bullet\,\,$,itemindent=12pt, leftmargin=*, itemsep=-7pt, topsep=-7pt}
\newlist{lista2}{itemize}{2}
\setlist[lista2]{label=$\circ$,itemindent=-8pt, leftmargin=30pt, itemsep=-7pt, topsep=-7pt}
\newlist{itemletras}{enumerate}{2}
\setlist[itemletras]{label=\alph*.$\;$, leftmargin=20pt, itemsep=-1pt, topsep=-7pt}
\setlength{\parskip}{5pt}
\hyphenpenalty=2000 \emergencystretch=20pt
\lhead{\textbf{ \niveldia{ } Problema \numero}}
\chead{\normalfont \ttfamily \archivo}
\rhead{\textbf {\contest}}
\newcommand*{\breakcolumn}{\null\vspace*{\fill}\columnbreak}

% Colores
\definecolor{gray}{rgb}{0.5, 0.5, 0.5}
\definecolor{pink}{rgb}{1, 0.25, 0.75}
\definecolor{dark green}{rgb}{0, 0.5, 0}
\definecolor{orange}{rgb}{1, 0.75, 0}
\definecolor{dark red}{rgb}{0.75, 0, 0}
\definecolor{purple}{rgb}{0.65, 0, 1}
\definecolor{dark blue}{rgb}{0, 0, 0.5}

% Espacios truchos
\def\ne{\vspace{-12pt}} % negative enter
\def\se{$ $\\}		    % simple enter
\def\de{\se\newline}	    % doble enter
\def\te{\de\newline}	    % triple enter
\def\tab{\hspace*{16pt}} % tab

% Separar en sílabas
\hyphenation{oia-no-va}

% Subrayado en color
% {\color{pink}\underline{\color{black}TEXTO}}\\

\begin{document}
%=============================MODIFICABLE=============================
%*********************************************************************
\def\contest{OIA Certamen Nacional}
\def\nombre{Estudiando Prefijos}
\def\author{Lautaro Lasorsa}
\def\niveldia{}
\def\numero{}
\def\archivo{prefijos}% Nombre del archivo
\def\version{}
%*********************************************************************
%*********************************************************************


\maketitle
\begin{multicols*}{2}

\section{Descripción del problema}

Lautaro quiere probar tus habilidades programando, y para eso te plantea el siguiente desafio:

Inicialmente, tienes $N$ palabras vacias (sin caracteres), númeradas de $1$ a $N$, y él te va a hacer $Q$ consultas / actualizaciones, que pueden ser de los siguientes tipos:

\begin{itemize}
    \item AGREGAR: Dado un entero $i$ y un caracter $c$ (letra minuscula del alfabeto ingles), agregar el caracter $c$ al final de palabra número $i$ ($1 \leq i \leq N$). 
    \item BORRAR: Dados dos enteros $i$ y $j$ ($1 \leq i \leq N$), borrar las últimas $j$ letras de la palabra número $i$. Se garantiza que la palabra tiene al menos $j$ letras (el resultado puede ser una palabra vacia)
    \item CAMBIAR: Dados dos enteros $i$ y $j$ ($1 \leq i , j \leq N$), intercambiar las palabras en dichas posiciones.
    \item COPIAR: Dados dos enteros $i$ y $j$ ($1 \leq i , j \leq N$), hacer que la palabra número $j$ se vuelva igual a la palabra número $i$.
    \item PREGUNTAR: Dados dos enteros $i$ y $j$ ($1 \leq i \leq j \leq N$), debes devolver la longitud del prefijo común más largo entre las palabras cuyos indices estén entre $i$ y $j$, inclusive. 
\end{itemize}

Para hacer el desafio más interesante, quiere que completes cada consulta / actualización antes de decirte la siguient.

\section{Detalles de implementación}
Debes implementar las funciones 
\begin{itemize}
    \item \texttt{Inicializar(N, Q)}, una función que será llamada una unica vez al inicio del programa, siendo los enteros \texttt{ENTERO} $N$ y $Q$ la cantidad de palabras y cantidad de consultas respectivamente. Por cada consulta se llamará a una de las siguientes funciones según corresponda.
    
    \item \texttt{Agregar(i,c)}, recibe el \texttt{ENTERO} $i$ y el \texttt{CARACTER} $c$ y debe implementar la actualización AGREGAR.
    
    \item \texttt{Borrar(i,j)}, recibe los \texttt{ENTERO} $i$ y  $j$ y debe implementar la actualización BORRAR.
    
    \item \texttt{Cambiar(i,j)}, recibe los \texttt{ENTERO} $i$ y  $j$ y debe implementar la actualización CAMBIAR.
    
    \item \texttt{Copiar(i,j)}, recibe los \texttt{ENTERO} $i$ y  $j$ y debe implementar la actualización COPIAR.
    
    \item \texttt{Preguntar(i,j)}, recibe los \texttt{ENTERO} $i$ y  $j$ y debe implementar la consulta PREGUNTAR. Debe devolver un \texttt{ENTERO} con la respuesta solicitada.
    
\end{itemize}

\section{Evaluador local}
El Evaluador Local primero leera una línea con los valore $N$ y $Q$. Luego llamará a \texttt{Inicializar(N,Q)}.

Posteriormente, leerá las $Q$ consultas / actualizaciones. Por cada una, leerá primero un entero $t$ que indica el tipo de consulta / actualización.

\begin{itemize}
    \item Si $t = 0$, leerá el \texttt{ENTERO} $i$ y el caracter $c$ y llamará a la función \texttt{Agregar(i,c)}
    
    \item Si $t = 1$, leerá los \texttt{ENTERO} $i$ y $j$ y llamrá a la función \texttt{Borrar(i,j)}
    
    \item Si $t = 2$, leerá los enteros \texttt{ENTERO} $i$ y  $j$ llamará a la función \texttt{Cambiar(i,j)}.
    
    \item Si $t = 3$, leerá los enteros \texttt{ENTERO} $i$ y  $j$ llamará a la función \texttt{Copiar(i,j)}.
    
    \item Si $t = 4$, leerá los enteros \texttt{ENTERO} $i$ y  $j$ llamará a la función \texttt{Preguntar(i,j)}, mostrando lo que esta devuelva por pantalla seguido de un fin de línea
    
\end{itemize}
\section{Cotas}
  \begin{itemize}
      \item  $1 \leq N, Q \leq 200.000$
  \end{itemize}

\section{Ejemplos}
Si el evaluador local recibe la siguiente entrada:

\caja{5.7cm}{\small
3 13 \\
0 1 a\\
0 1 a\\
0 1 a\\
4 1 3\\
0 2 b\\
0 3 a\\
0 3 a\\
0 3 a\\
4 1 2\\
2 2 3\\
4 1 2\\
3 2 3\\
4 1 3\\
1 2 2\\
4 1 3
}

Una  implementación  correcta  podrá devolver:

\caja{5.7cm}{\small
0\\
0\\
3\\
3\\
1
}

En cambio, si recibe:

\caja{5.7cm}{\small
3 19\\
0 1 a\\
0 1 b\\
0 1 c\\
0 1 d\\
0 1 e\\
4 1 2\\
0 2 a\\
4 1 2\\
0 2 b\\
4 1 2\\
0 2 c\\
4 1 2\\
0 2 d\\
4 1 2\\
0 2 e\\
4 1 2\\
4 1 3\\
3 1 3\\
4 1 3
}

Podrá devolver:

\caja{5.7cm}{\small
0\\
1\\
2\\
3\\
4\\
5\\
0\\
5
}
\section{Subtareas}

\begin{enumerate}
  \item $1 \leq N, Q \leq 3.000$ y no hay llamados a la función \texttt{Copiar(i,j)}. (10 puntos)
  \item $1 \leq N \leq 200.000$ y $1 \leq Q \leq 3.000$. (25 puntos)
  \item Hay menos de 100 llamados a la función \texttt{Preguntar(i,j)}, y en todos los llamados a la función \texttt{Borrar(i,j)} $j = 1$ (15 puntos)
  \item Hay menos de 100 llamados a la función \texttt{Preguntar(i,j)} (10 puntos)
  \item Sin restricciones adicionales. (40 puntos)
\end{enumerate}

\end{multicols*}
\end{document}
