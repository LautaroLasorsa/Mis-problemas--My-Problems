% Plantilla para problemas de OIA
% por Carolina González y Martín Villagra
% (hecho a ojo)

\documentclass[fontsize=13pt, paper=a4, DIV=calc]{scrartcl}
\usepackage{silence}
\WarningFilter{scrartcl}{Usage of package `fancyhdr'}
\WarningFilter{scrartcl}{Usage of package `titlesec'}
\WarningFilter{scrartcl}{Activating an ugly workaround for a missing}
\WarningFilter{titlesec}{Non standard sectioning command detected}
\usepackage[spanish]{babel}
\usepackage[utf8]{inputenc}
\usepackage[pdftex]{graphicx}
\usepackage{amsmath,amssymb,amsfonts,latexsym,cancel}
\usepackage{color}
\usepackage{rotating}
\usepackage{array}
\usepackage{multirow}
\usepackage{multicol}
\usepackage{pbox}
\usepackage{tikz}
\usepackage{makeidx}
\usepackage{eso-pic}
\usepackage{lastpage}
\usepackage{fancyhdr}
\usepackage{listings}
\usepackage{titlesec}
\usepackage{enumitem}
\usepackage{helvet}
\usepackage[T1]{fontenc}
\usepackage[helvet]{sfmath}
\everymath={\mathsf\mathbf}
\decimalpoint
\textheight = 26cm
\textwidth = 18cm
\topmargin = -1.5cm
\oddsidemargin = -1.5cm
\hoffset = 0.55cm
\fancyheadoffset{-1pt}
\pagestyle{fancy}
\lfoot{\scriptsize\version}
\cfoot{}
\rfoot{\scriptsize hoja {\thepage} de \pageref{LastPage}}
\renewcommand{\headrule}{\vspace*{-9pt}\rule{\textwidth}{0.2pt}\\[\dimexpr-\baselineskip+0.5pt]\rule{\textwidth}{0.2pt}}
\renewcommand{\footrulewidth}{0.2pt}
\setlength{\footskip}{20pt}
\AddToShipoutPicture{
	\begin{tikzpicture}
		\draw[white] (0,0) rectangle (0,0);
		\draw[very thick] (0.3,0.8) rectangle (19.45,28.9);
	\end{tikzpicture}
}
\renewcommand{\familydefault}{\sfdefault}
\setlength{\parindent}{0.5cm}
\setlength{\columnsep}{30pt}
\setlength{\columnseprule}{0.2pt}
\setlength{\headsep}{0.1in}
\titleformat{\section} {\normalfont}{}{0pt}{\textbf}
\titlespacing*{\section}{0pt}{10pt}{0pt}
\newcommand{\caja}[2]{
	{\fontfamily{pcr}\selectfont\large
	\hspace*{16pt}{\setlength{\fboxrule}{0pt}\fboxsep4pt\fbox{\pbox{\textwidth}{
		\hspace*{16pt}{\setlength{\fboxrule}{0.5pt}\fboxsep8pt\framebox[#1][l]{\pbox{\textwidth}{#2}}}
	}}}
	}\\\par
}
\renewcommand*{\maketitle}{
	\begin{center}
		\textbf{\nombre}\\
		\textit{\footnotesize Contribución de \author}
	\end{center}
}
\newlist{lista}{itemize}{2}
\setlist[lista]{label=$\bullet\,\,$,itemindent=12pt, leftmargin=*, itemsep=-7pt, topsep=-7pt}
\newlist{lista2}{itemize}{2}
\setlist[lista2]{label=$\circ$,itemindent=-8pt, leftmargin=30pt, itemsep=-7pt, topsep=-7pt}
\newlist{itemletras}{enumerate}{2}
\setlist[itemletras]{label=\alph*.$\;$, leftmargin=20pt, itemsep=-1pt, topsep=-7pt}
\setlength{\parskip}{5pt}
\hyphenpenalty=2000 \emergencystretch=20pt
\lhead{\textbf{ \niveldia{ } Problema \numero}}
\chead{\normalfont \ttfamily \archivo}
\rhead{\textbf {\contest}}
\newcommand*{\breakcolumn}{\null\vspace*{\fill}\columnbreak}

% Colores
\definecolor{gray}{rgb}{0.5, 0.5, 0.5}
\definecolor{pink}{rgb}{1, 0.25, 0.75}
\definecolor{dark green}{rgb}{0, 0.5, 0}
\definecolor{orange}{rgb}{1, 0.75, 0}
\definecolor{dark red}{rgb}{0.75, 0, 0}
\definecolor{purple}{rgb}{0.65, 0, 1}
\definecolor{dark blue}{rgb}{0, 0, 0.5}

% Espacios truchos
\def\ne{\vspace{-12pt}} % negative enter
\def\se{$ $\\}		    % simple enter
\def\de{\se\newline}	    % doble enter
\def\te{\de\newline}	    % triple enter
\def\tab{\hspace*{16pt}} % tab

% Separar en sílabas
\hyphenation{oia-no-va}

% Subrayado en color
% {\color{pink}\underline{\color{black}TEXTO}}\\

\begin{document}
%=============================MODIFICABLE=============================
%*********************************************************************
\def\contest{OIA Certamen Nacional}
\def\nombre{Operando Circuitos}
\def\author{Lautaro Lasorsa}
\def\niveldia{}
\def\numero{}
\def\archivo{circuitos}% Nombre del archivo
\def\version{}
%*********************************************************************
%*********************************************************************


\maketitle
\begin{multicols*}{2}

\section{Descripción del problema}

Agustín tiene un circuito electronico, el cuál está compuesto por compuertas de entrada, diversas compuertas logicas y luces LED. En total tiene $N$ componentes.

Cada compuerta tiene una salida, que puede estar conectada a la entrada de otra compuerta o a una luz LED.

En cada segundo, él puede elegir algunas de las compuertas de entrada y enviar corriente por ellas, corriente que pasa hasta la salida de la compuerta.

Cada compuerta logica tiene exactamente 2 entradas, que serán las salidas de otras compuertas,

\begin{itemize}
    \item Compuertas AND: Estas compuertas van a enviar corriente por su salida solo si reciben corriente por sus 2 entradas simultaneamente.
    \item Compuertas OR: Estas compuertas van a enviar corriente por su salida si reciben corriente por al menos una de sus entradas.
    \item Compuertas XOR: Estas compuertas van a enviar corriente por su salida si reciben corriente alguna de sus entradas, pero no por las 2 simultaneamente.
\end{itemize}

Agustín quiere saber, para cada luz LED, qué segundos va a estar encendida. Las luces LED se encienden cuando reciben electricidad por su unica entrada.

La activación de las compuertas y la transmisión de energia dentro del circuito es instantanea.

\section{Detalles de implementación}
Debes implementar la función \texttt{Circuitos(tipo,input)}, siendo $tipo$ un \texttt{CADENA DE CARACTER}, donde $tipo_i$ indica el tipo del $i-esimo$ componente, de la siguiente manera:
\begin{itemize}
    \item $E$: Compuerta de entrada
    \item $O$: Compuerta OR
    \item $A$: Compuerta AND
    \item $X$: Compuerta XOR
    \item $L$: Luz LED
\end{itemize}
Por su parte $input$ es un \texttt{VECTOR DE VECTOR DE ENTERO}. El contenido de $input_i$ dependerá del tipo del componente $i-esimo$:
\begin{itemize}
    \item Si es una compuerta de entrada, $input_i$ indicará los segundos en los cuales se enviará corriente por esa entrada.
    \item Si es una compuerta AND, OR o XOR, $input_i$ contendrá 2 elementos, que serán los indices de las compuertas a cuyas salidas estan conectadas las entradas de esta.
    \item Si es una luz LED, contendra un unico entero que será el índice de la compuerta a cuya salida está conectada la entrada de la luz LED.
\end{itemize}
Notar que los indices de los componentes se indican en base 0, por lo que van de $0$ a $N-1$.

Ambos vectores son de longitud $N$.

Notar que en el circuito que armo Agustín no hay ciclos, es decir, en ningún caso la salida de una compuerta está directa o indirectametne conectada a una de las entradas de la misma. 

La función deberá retornar un \texttt{VECTOR DE VECTOR DE ENTEROS}, que en la $i-esima$ posición contendra un vector vacio si ese componente no es una luz LED, y en el caso de que sea una luz LED un vector con los segundos en los cuales esa luz estará prendida, en cualquier orden. (sin repetidos)

\section{Evaluador local}
El evaluador local leera primero el entero $N$.
Luego recibirá en una línea la cadena $tipo$, que consiste de $N$ caracteres separados por un espacio entre si.
En las siguientes $N$ líneas leera el vector $input$
En la $(i+2)-esima$ línea leera primero $|input_i|$, y luego los $|input_i|$ elementos del vector $input_i$

Posteriormente llamará a la funcion \texttt{Circuitos(tipo,input)}, y mostrara en $N$ líneas lo que devuelva la función. Una línea para cada uno de los vectores.

\section{Cotas}
  \begin{itemize}
     \item $ 1 \leq N \leq 100.000$
     \item Siendo $M$ la suma de las longitudes de los vectores $input_i$.
     $1 \leq M \leq 300.000$
     \item Si $tipo_i = E$, $1 \leq input_{i,j}  \leq 10^9 $ para todo $0 \leq j < |input_i|$
  \end{itemize}

\section{Ejemplos}
Si el evaluador local recibe la siguiente entrada:

\caja{5.7cm}{\small
12\\
EEALEEOLEEXL\\
5 1 2 3 4 5\\
3 3 1 5\\
2 0 1\\
1 2\\
5 11 12 10 9 99\\
3 13 11 9\\
2 4 5\\
1 6\\
10 1 2 3 4 5 6 7 8 9\\
5 1 3 5 7 9\\
2 8 9\\
1 10
}

Una  implementación  correcta  deberá devolver:

\caja{5.7cm}{\small
\hfill\break
\hfill\break
\hfill\break
1 3 5\\
\hfill\break
\hfill\break
\hfill\break
9 10 11 12 13 99\\
\hfill\break
\hfill\break
\hfill\break
2 4 6 8
}
Notar que si $tipo_i \neq L$ la $i-esima$ línea es dejada en blanco.

En cambio, si recibe:

\caja{5.7cm}{\small
16\\
EEEEEEEEAAAAOOXL\\
5 1 2 3 4 5\\
3 3 4 5\\
7 1 3 5 7 9 11 13\\
3 1 4 9\\
5 10 11 12 13 14\\
3 2 4 6\\
5 2 4 8 16 32\\
3 2 8 32\\
2 0 1\\
2 2 3\\
2 4 5\\
2 6 7\\
2 8 9\\
2 10 11\\
2 12 13\\
1 14
}

Deberá devolver:

\caja{5.7cm}{\small
\hfill\break
\hfill\break
\hfill\break
\hfill\break
\hfill\break
\hfill\break
\hfill\break
\hfill\break
\hfill\break
\hfill\break
\hfill\break
\hfill\break
\hfill\break
\hfill\break
\\1 3 4 5 8 9 32
}
\section{Subtareas}

\begin{enumerate}
  \item $1 \leq N, M \leq 500$. (10 puntos)
  \item No hay compuertas OR ni XOR  (10 puntos)
  \item No hay compuertas AND ni XOR (10 puntos)
  \item No hay compuertas ANd ni OR (10 puntos)
  \item Si $tipo_i = E$, $1 \leq input_{i,j}  \leq 5.000 $ para todo $0 \leq j < |input_i|$ (20 puntos)
  \item Sin restricciones adicionales (40 puntos) 
  
\end{enumerate}

\end{multicols*}
\end{document}
