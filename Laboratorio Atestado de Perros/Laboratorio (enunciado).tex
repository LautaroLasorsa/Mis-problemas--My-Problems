% Plantilla para problemas de OIA
% por Carolina González y Martín Villagra
% (hecho a ojo)

\documentclass[fontsize=13pt, paper=a4, DIV=calc]{scrartcl}
\usepackage{silence}
\WarningFilter{scrartcl}{Usage of package `fancyhdr'}
\WarningFilter{scrartcl}{Usage of package `titlesec'}
\WarningFilter{scrartcl}{Activating an ugly workaround for a missing}
\WarningFilter{titlesec}{Non standard sectioning command detected}
\usepackage[spanish]{babel}
\usepackage[utf8]{inputenc}
\usepackage[pdftex]{graphicx}
\usepackage{amsmath,amssymb,amsfonts,latexsym,cancel}
\usepackage{color}
\usepackage{rotating}
\usepackage{array}
\usepackage{multirow}
\usepackage{multicol}
\usepackage{pbox}
\usepackage{tikz}
\usepackage{makeidx}
\usepackage{eso-pic}
\usepackage{lastpage}
\usepackage{fancyhdr}
\usepackage{listings}
\usepackage{titlesec}
\usepackage{enumitem}
\usepackage{helvet}
\usepackage[T1]{fontenc}
\usepackage[helvet]{sfmath}
\everymath={\mathsf\mathbf}
\decimalpoint
\textheight = 26cm
\textwidth = 18cm
\topmargin = -1.5cm
\oddsidemargin = -1.5cm
\hoffset = 0.55cm
\fancyheadoffset{-1pt}
\pagestyle{fancy}
\lfoot{\scriptsize\version}
\cfoot{}
\rfoot{\scriptsize hoja {\thepage} de \pageref{LastPage}}
\renewcommand{\headrule}{\vspace*{-9pt}\rule{\textwidth}{0.2pt}\\[\dimexpr-\baselineskip+0.5pt]\rule{\textwidth}{0.2pt}}
\renewcommand{\footrulewidth}{0.2pt}
\setlength{\footskip}{20pt}
\AddToShipoutPicture{
	\begin{tikzpicture}
		\draw[white] (0,0) rectangle (0,0);
		\draw[very thick] (0.3,0.8) rectangle (19.45,28.9);
	\end{tikzpicture}
}
\renewcommand{\familydefault}{\sfdefault}
\setlength{\parindent}{0.5cm}
\setlength{\columnsep}{30pt}
\setlength{\columnseprule}{0.2pt}
\setlength{\headsep}{0.1in}
\titleformat{\section} {\normalfont}{}{0pt}{\textbf}
\titlespacing*{\section}{0pt}{10pt}{0pt}
\newcommand{\caja}[2]{
	{\fontfamily{pcr}\selectfont\large
	\hspace*{16pt}{\setlength{\fboxrule}{0pt}\fboxsep4pt\fbox{\pbox{\textwidth}{
		\hspace*{16pt}{\setlength{\fboxrule}{0.5pt}\fboxsep8pt\framebox[#1][l]{\pbox{\textwidth}{#2}}}
	}}}
	}\\\par
}
\renewcommand*{\maketitle}{
	\begin{center}
		\textbf{\nombre}\\
		\textit{\footnotesize Contribución de \author}
	\end{center}
}
\newlist{lista}{itemize}{2}
\setlist[lista]{label=$\bullet\,\,$,itemindent=12pt, leftmargin=*, itemsep=-7pt, topsep=-7pt}
\newlist{lista2}{itemize}{2}
\setlist[lista2]{label=$\circ$,itemindent=-8pt, leftmargin=30pt, itemsep=-7pt, topsep=-7pt}
\newlist{itemletras}{enumerate}{2}
\setlist[itemletras]{label=\alph*.$\;$, leftmargin=20pt, itemsep=-1pt, topsep=-7pt}
\setlength{\parskip}{5pt}
\hyphenpenalty=2000 \emergencystretch=20pt
\lhead{\textbf{ \niveldia{ } Problema \numero}}
\chead{\normalfont \ttfamily \archivo}
\rhead{\textbf {\contest}}
\newcommand*{\breakcolumn}{\null\vspace*{\fill}\columnbreak}

% Colores
\definecolor{gray}{rgb}{0.5, 0.5, 0.5}
\definecolor{pink}{rgb}{1, 0.25, 0.75}
\definecolor{dark green}{rgb}{0, 0.5, 0}
\definecolor{orange}{rgb}{1, 0.75, 0}
\definecolor{dark red}{rgb}{0.75, 0, 0}
\definecolor{purple}{rgb}{0.65, 0, 1}
\definecolor{dark blue}{rgb}{0, 0, 0.5}

% Espacios truchos
\def\ne{\vspace{-12pt}} % negative enter
\def\se{$ $\\}		    % simple enter
\def\de{\se\newline}	    % doble enter
\def\te{\de\newline}	    % triple enter
\def\tab{\hspace*{16pt}} % tab

% Separar en sílabas
\hyphenation{oia-no-va}

% Subrayado en color
% {\color{pink}\underline{\color{black}TEXTO}}\\

\begin{document}
%=============================MODIFICABLE=============================
%*********************************************************************
\def\contest{OIA Certamen Nacional}
\def\nombre{Laboratorio Atestado de Perros}
\def\author{Lautaro Lasorsa}
\def\niveldia{}
\def\numero{}
\def\archivo{laboratorio}% Nombre del archivo
\def\version{}
%*********************************************************************
%*********************************************************************


\maketitle
\begin{multicols*}{2}

\section{Descripción del problema}

Durante el Training Camp Argentina 2019, realizado en Córdoba, fue necesario utilizar el laboratorio de informatica para realizar competencias. El problema es que en la UNC hay muchos perros dando vueltas, y los perros fueron entrando al laboratorio y obstruyendo el paso.

Más especificamente, el laboratorio puede representarse como una grilla de $N x M$ baldosas. Una baldosa puede estar libre, es decir se puede trancitar sobre ella, o puede haber mobiliario (escritorios para computadoras principalmente) sobre ella, que no puede ser movido e impide trancitarla. 

Es importante marcar que el laboratorio tiene una unica entrada, pero que puede abarcar varias baldosas.

Se dice que una baldosa libre es accesible si es posible armar un camino que pase solo por baldosas libres, pudiendo moverse desde una mandosa solo hacia arriba, abajo, izquierda y derecha, que una la entrada con dicha baldosa. Normalmente (cuando no hay perros), todas las baldosas libres son accesibles.

Además, han ingresado, todos en momentos distintos, $P$ perros, de la siguiente forma : 
\begin{itemize}
    \item Cuando un perro entra, se recuesta en una baldosa libre, bloqueandola (no es posible transitar sobre una baldosa en la que hay un perro). Como está muy comodo ahí permanecerá indefinidamente en ese lugar.
    \item Los perros son agiles y además se dejan pasar entre ellos, por lo que un perro podrá recostarse en una baldosa incluso si esta no es accesible para las personas. 
\end{itemize}

Dados los lugares en los que se recosto cada perro, en el orden en que fueron entrando, se desea saber cuantas baldosas accesibles había después de la entrada de cada perro. 


\section{Detalles de implementación}
Se debe implementar la funcion \texttt{laboratorio(labo,perros)}
\begin{itemize}
    \item \texttt{labo} es un vector de largo $N$ de cadenas de caracteres de largo $M$, donde la posicion \texttt{labos[i][j]} describe la baldosa en fila $i$ y columna $j$. Su valor será :
    \begin{itemize}
        \item'.' si es una baldosa libre.
        \item '\#' si es una baldosa con mobiliario.
        \item 'E' si es una baldosa aledaña a la entrada. Todas las baldosas 'E' estan en el borden del laboratorio, y son aledañas entre si. Las baldosas 'E' son, a su vez, baldosas libres.
    \end{itemize}
    \item \texttt{perros} es un vector de largo $P$ de pares de enteros, donde el par \texttt{perros[i]} indica en que baldosa se ubicara el $i-esimo$ perro. 
\end{itemize}

La función debe retornar un vector de enteros de largo $P$, donde la $i-esima$ posición contenga la cantidad de baldosas libres accesibl es después de que entrase el $i-esimo$ perro. 

\columnbreak

\section{Evaluador local}

El evaluador local lee de la entrada estándar: 
\begin{itemize}
    \item Una línea con los valores $N$, $M$ y $P$
    \item $N$ líneas de $M$ caracteres cada una, sin espacios, describiendo el estado inicial del laboratorio. (el contenido de \texttt{labo})
    \item Luego $P$ líneas, cada una con 2 enteros. El primero entre 0 y $N-1$, y el segundo entre 0 y $M-1$, son los valores de \texttt{perros[i]}
\end{itemize}

Devuelve por la salida estándar una línea con los valores del vector retornado por \texttt{laboratorio}.

\section{Cotas}
  \begin{itemize}
      \item $1 \leq NxM \leq 1.000.000$
      \item $0 \leq P \leq 100.000$
  \end{itemize}

\section{Ejemplos}
Si el evaluador local recibe la siguiente entrada:

\caja{5.7cm}{\small
4 4 5\\
...\#\\
\#...\\
E..\#\\
E..\#\\
1 3\\
0 0\\
0 1\\
2 0\\
3 0
}

Una  implementación  correcta  deberá devolver:

\caja{5.7cm}{\small
11 10 9 8 0
}

Si en cambio recibe:

\caja{5.7cm}{\small
1 10 5\\
\#....E...\#\\
0 2\\
0 1\\
0 5\\
0 6\\
0 7
}

Una implementación correcta devolverá:

\caja{5.7cm}{\small
6 6 0 0 0
}

Y si en cambio recibe :

\caja{5.7cm}{\small
10 10 6\\
E..\#...\#..\\
E..\#...\#..\\
E..\#...\#..\\ 
E..\#\#.\#\#.\#\\ 
E.........\\  
E..\#\#.\#\#.\#\\ 
E..\#...\#..\\ 
E..\#...\#..\\ 
E..\#...\#..\\ 
E..\#...\#..\\ 
3 5\\
5 5\\
3 8\\
5 8\\
4 7\\
4 3
}

Devolverá:

\caja{5.7cm}{\small
66 53 46 37 34 30
}
\columnbreak

\section{Subtareas}

\begin{enumerate}
    \item $P=1$ (10 puntos)
    \item $NxMxP \leq 1.000.000$ (20 puntos)
    \item $N=1$ (20 puntos)
    \item Sin restricciones adicionales (50 puntos)
\end{enumerate}

\end{multicols*}
\end{document}
