% Plantilla para problemas de OIA
% por Carolina González y Martín Villagra
% (hecho a ojo)

\documentclass[fontsize=13pt, paper=a4, DIV=calc]{scrartcl}
\usepackage{silence}
\WarningFilter{scrartcl}{Usage of package `fancyhdr'}
\WarningFilter{scrartcl}{Usage of package `titlesec'}
\WarningFilter{scrartcl}{Activating an ugly workaround for a missing}
\WarningFilter{titlesec}{Non standard sectioning command detected}
\usepackage[spanish]{babel}
\usepackage[utf8]{inputenc}
\usepackage[pdftex]{graphicx}
\usepackage{amsmath,amssymb,amsfonts,latexsym,cancel}
\usepackage{color}
\usepackage{rotating}
\usepackage{array}
\usepackage{multirow}
\usepackage{multicol}
\usepackage{pbox}
\usepackage{tikz}
\usepackage{makeidx}
\usepackage{eso-pic}
\usepackage{lastpage}
\usepackage{fancyhdr}
\usepackage{listings}
\usepackage{titlesec}
\usepackage{enumitem}
\usepackage{helvet}
\usepackage[T1]{fontenc}
\usepackage[helvet]{sfmath}
\everymath={\mathsf\mathbf}
\decimalpoint
\textheight = 26cm
\textwidth = 18cm
\topmargin = -1.5cm
\oddsidemargin = -1.5cm
\hoffset = 0.55cm
\fancyheadoffset{-1pt}
\pagestyle{fancy}
\lfoot{\scriptsize\version}
\cfoot{}
\rfoot{\scriptsize hoja {\thepage} de \pageref{LastPage}}
\renewcommand{\headrule}{\vspace*{-9pt}\rule{\textwidth}{0.2pt}\\[\dimexpr-\baselineskip+0.5pt]\rule{\textwidth}{0.2pt}}
\renewcommand{\footrulewidth}{0.2pt}
\setlength{\footskip}{20pt}
\AddToShipoutPicture{
	\begin{tikzpicture}
		\draw[white] (0,0) rectangle (0,0);
		\draw[very thick] (0.3,0.8) rectangle (19.45,28.9);
	\end{tikzpicture}
}
\renewcommand{\familydefault}{\sfdefault}
\setlength{\parindent}{0.5cm}
\setlength{\columnsep}{30pt}
\setlength{\columnseprule}{0.2pt}
\setlength{\headsep}{0.1in}
\titleformat{\section} {\normalfont}{}{0pt}{\textbf}
\titlespacing*{\section}{0pt}{10pt}{0pt}
\newcommand{\caja}[2]{
	{\fontfamily{pcr}\selectfont\large
	\hspace*{16pt}{\setlength{\fboxrule}{0pt}\fboxsep4pt\fbox{\pbox{\textwidth}{
		\hspace*{16pt}{\setlength{\fboxrule}{0.5pt}\fboxsep8pt\framebox[#1][l]{\pbox{\textwidth}{#2}}}
	}}}
	}\\\par
}
\renewcommand*{\maketitle}{
	\begin{center}
		\textbf{\nombre}\\
		\textit{\footnotesize Contribución de \author}
	\end{center}
}
\newlist{lista}{itemize}{2}
\setlist[lista]{label=$\bullet\,\,$,itemindent=12pt, leftmargin=*, itemsep=-7pt, topsep=-7pt}
\newlist{lista2}{itemize}{2}
\setlist[lista2]{label=$\circ$,itemindent=-8pt, leftmargin=30pt, itemsep=-7pt, topsep=-7pt}
\newlist{itemletras}{enumerate}{2}
\setlist[itemletras]{label=\alph*.$\;$, leftmargin=20pt, itemsep=-1pt, topsep=-7pt}
\setlength{\parskip}{5pt}
\hyphenpenalty=2000 \emergencystretch=20pt
\lhead{\textbf{ \niveldia{ } Problema \numero}}
\chead{\normalfont \ttfamily \archivo}
\rhead{\textbf {\contest}}
\newcommand*{\breakcolumn}{\null\vspace*{\fill}\columnbreak}

% Colores
\definecolor{gray}{rgb}{0.5, 0.5, 0.5}
\definecolor{pink}{rgb}{1, 0.25, 0.75}
\definecolor{dark green}{rgb}{0, 0.5, 0}
\definecolor{orange}{rgb}{1, 0.75, 0}
\definecolor{dark red}{rgb}{0.75, 0, 0}
\definecolor{purple}{rgb}{0.65, 0, 1}
\definecolor{dark blue}{rgb}{0, 0, 0.5}

% Espacios truchos
\def\ne{\vspace{-12pt}} % negative enter
\def\se{$ $\\}		    % simple enter
\def\de{\se\newline}	    % doble enter
\def\te{\de\newline}	    % triple enter
\def\tab{\hspace*{16pt}} % tab

% Separar en sílabas
\hyphenation{oia-no-va}

% Subrayado en color
% {\color{pink}\underline{\color{black}TEXTO}}\\

\begin{document}
%=============================MODIFICABLE=============================
%*********************************************************************
\def\contest{Certamen Jurisdiccional OIA 2022}
\def\nombre{Formando equipos impares}
\def\author{Lautaro Lasorsa}
\def\niveldia{Nivel 3}
\def\numero{4}
\def\archivo{impares}% Nombre del archivo
\def\version{}
%*********************************************************************
%*********************************************************************


\maketitle
\begin{multicols*}{2}

\section{Descripción del problema}

Una cierta empresa, que se organiza de forma jerárquica, desea armar equipos para llevar a cabo diferentes proyectos.

La empresa tiene $N$ empleados, numerados desde $0$ hasta $N-1$ inclusive. Está dirigida por un gerente general, que es supervisor directo de algunos empleados, que son a su vez supervisores directos de otros empleados de menor jerarquía, y así siguiendo hasta llegar a aquellos empleados sin subordinados.

En esta empresa cada empleado (excepto el gerente general) tiene \textbf{exactamente un supervisor directo}, y puede supervisar a 0 o más empleados. Todos los empleados son directa o indirectamente supervisados por el gerente general.

Un equipo deberá estar formado por:
\begin{itemize}
    \item Un empleado designado como \textit{director} del equipo.
    \item Opcionalmente, puede haber más empleados además del director, que sean directa o indirectamente supervisados por él.
    \item Es un requisito que por cada miembro del equipo que no sea el director del equipo, \textbf{su supervisor directo también forme parte del equipo}.
    \item Para evitar empates en las votaciones y así facilitar la toma de decisiones, los equipos deben estar conformados por \textbf{una cantidad impar} de empleados.
\end{itemize}

Siguiendo estas reglas, se debe calcular cuántas maneras diferentes existen de formar un equipo eligiendo algunos empleados de la empresa.

Como este número puede ser muy grande, se debe indicar el resto al dividirlo por $10^9+7$

\columnbreak

\section{Detalles de implementación}

Debes implementar la función \texttt{impares(subordinados)}, que recibe un único parámetro \texttt{subordinados}: un arreglo de $N$ arreglos de enteros. El arreglo \texttt{subordinados[i]} con $0 \leq i < N$ indica los \textit{subordinados} del empleado $i$, es decir, aquellos empleados que tienen al $i$ como supervisor directo.
La función debe retornar un entero con el valor solicitado.

El gerente general es el empleado 0. 


\section{Evaluador local}

El evaluador lee de la entrada estándar: 
\begin{itemize}
    \item Una línea con el entero $N$
    \item $N$ líneas, que representan los arreglos \texttt{subordinados[i]} en orden, para $i$ entre $0$ y $N-1$ inclusive. Cada línea comienza con un número $k$ que indica el tamaño del correspondiente arreglo \texttt{subordinados[i]}, seguido de los $k$ elementos del arreglo.
\end{itemize}

Escribe en la salida estándar una línea con el resultado retornado por la llamada \texttt{impares(subordinados)}.

\section{Restricciones}
  \begin{itemize}
      \item $1 \leq N \leq 200.000$
  \end{itemize}

\columnbreak

\section{Ejemplos}
{\footnotesize
Para la siguiente entrada:

\caja{3.3cm}{\small
6\\
5 1 2 3 4 5\\
0\\
0\\
0\\
0\\
0
}

La respuesta correcta es:

\caja{1.2cm}{\small
21
}

En cambio para:

\caja{1.3cm}{\small
5\\
1 1\\
1 2\\
1 3\\
1 4\\
0
}

La respuesta es:

\caja{1.2cm}{\small
9
}

Y para:

\caja{2.2cm}{\small
15\\
2 1 2\\
2 3 4\\
2 5 6\\
2 7 8\\
2 9 10\\
2 11 12\\
2 13 14\\
0\\
0\\
0\\
0\\
0\\
0\\
0\\
0
}

La respuesta es:

\vspace{-0.1cm}\caja{1.3cm}{\small
380
}
}

\section{Subtareas}

\small

\begin{enumerate}
    \item $N \leq 10$ (12 puntos)
    \item $\texttt{subordinados[i]}$ tiene como máximo un elemento para $0 \leq i < N$ (12 puntos)
    
    \item $\texttt{subordinados[0]}$ tiene tamaño 2, y $\texttt{subordinados[i]}$ tiene como máximo un elemento para $1 \leq i < N$ (6 puntos)
    
    \item $\texttt{subordinados[0]}$ tiene hasta $20$ elementos, $\texttt{subordinados[i]}$ tiene a lo más un elemento si $1 \leq i < N$ (15 puntos)
    
    \item $\texttt{subordinados[i]}$ tiene hasta dos elementos para $0 \leq i < N$ (15 puntos)
    \item Sin más restricción (40 puntos)
\end{enumerate}

\end{multicols*}
\end{document}
