% Plantilla para problemas de OIA
% por Carolina González y Martín Villagra
% (hecho a ojo)

\documentclass[fontsize=13pt, paper=a4, DIV=calc]{scrartcl}
\usepackage{silence}
\WarningFilter{scrartcl}{Usage of package `fancyhdr'}
\WarningFilter{scrartcl}{Usage of package `titlesec'}
\WarningFilter{scrartcl}{Activating an ugly workaround for a missing}
\WarningFilter{titlesec}{Non standard sectioning command detected}
\usepackage[spanish]{babel}
\usepackage[utf8]{inputenc}
\usepackage[pdftex]{graphicx}
\usepackage{amsmath,amssymb,amsfonts,latexsym,cancel}
\usepackage{color}
\usepackage{rotating}
\usepackage{array}
\usepackage{multirow}
\usepackage{multicol}
\usepackage{pbox}
\usepackage{tikz}
\usepackage{makeidx}
\usepackage{eso-pic}
\usepackage{lastpage}
\usepackage{fancyhdr}
\usepackage{listings}
\usepackage{titlesec}
\usepackage{enumitem}
\usepackage{helvet}
\usepackage[T1]{fontenc}
\usepackage[helvet]{sfmath}
\everymath={\mathsf\mathbf}
\decimalpoint
\textheight = 26cm
\textwidth = 18cm
\topmargin = -1.5cm
\oddsidemargin = -1.5cm
\hoffset = 0.55cm
\fancyheadoffset{-1pt}
\pagestyle{fancy}
\lfoot{\scriptsize\version}
\cfoot{}
\rfoot{\scriptsize hoja {\thepage} de \pageref{LastPage}}
\renewcommand{\headrule}{\vspace*{-9pt}\rule{\textwidth}{0.2pt}\\[\dimexpr-\baselineskip+0.5pt]\rule{\textwidth}{0.2pt}}
\renewcommand{\footrulewidth}{0.2pt}
\setlength{\footskip}{20pt}
\AddToShipoutPicture{
	\begin{tikzpicture}
		\draw[white] (0,0) rectangle (0,0);
		\draw[very thick] (0.3,0.8) rectangle (19.45,28.9);
	\end{tikzpicture}
}
\renewcommand{\familydefault}{\sfdefault}
\setlength{\parindent}{0.5cm}
\setlength{\columnsep}{30pt}
\setlength{\columnseprule}{0.2pt}
\setlength{\headsep}{0.1in}
\titleformat{\section} {\normalfont}{}{0pt}{\textbf}
\titlespacing*{\section}{0pt}{10pt}{0pt}
\newcommand{\caja}[2]{
	{\fontfamily{pcr}\selectfont\large
	\hspace*{16pt}{\setlength{\fboxrule}{0pt}\fboxsep4pt\fbox{\pbox{\textwidth}{
		\hspace*{16pt}{\setlength{\fboxrule}{0.5pt}\fboxsep8pt\framebox[#1][l]{\pbox{\textwidth}{#2}}}
	}}}
	}\\\par
}
\renewcommand*{\maketitle}{
	\begin{center}
		\textbf{\nombre}\\
		\textit{\footnotesize Contribución de \author}
	\end{center}
}
\newlist{lista}{itemize}{2}
\setlist[lista]{label=$\bullet\,\,$,itemindent=12pt, leftmargin=*, itemsep=-7pt, topsep=-7pt}
\newlist{lista2}{itemize}{2}
\setlist[lista2]{label=$\circ$,itemindent=-8pt, leftmargin=30pt, itemsep=-7pt, topsep=-7pt}
\newlist{itemletras}{enumerate}{2}
\setlist[itemletras]{label=\alph*.$\;$, leftmargin=20pt, itemsep=-1pt, topsep=-7pt}
\setlength{\parskip}{5pt}
\hyphenpenalty=2000 \emergencystretch=20pt
\lhead{\textbf{ \niveldia{ } Problema \numero}}
\chead{\normalfont \ttfamily \archivo}
\rhead{\textbf {\contest}}
\newcommand*{\breakcolumn}{\null\vspace*{\fill}\columnbreak}

% Colores
\definecolor{gray}{rgb}{0.5, 0.5, 0.5}
\definecolor{pink}{rgb}{1, 0.25, 0.75}
\definecolor{dark green}{rgb}{0, 0.5, 0}
\definecolor{orange}{rgb}{1, 0.75, 0}
\definecolor{dark red}{rgb}{0.75, 0, 0}
\definecolor{purple}{rgb}{0.65, 0, 1}
\definecolor{dark blue}{rgb}{0, 0, 0.5}

% Espacios truchos
\def\ne{\vspace{-12pt}} % negative enter
\def\se{$ $\\}		    % simple enter
\def\de{\se\newline}	    % doble enter
\def\te{\de\newline}	    % triple enter
\def\tab{\hspace*{16pt}} % tab

% Separar en sílabas
\hyphenation{oia-no-va}

% Subrayado en color
% {\color{pink}\underline{\color{black}TEXTO}}\\

\begin{document}
%=============================MODIFICABLE=============================
%*********************************************************************
\def\contest{OIA Certamen Nacional Nivel 3}
\def\nombre{Jugando con Divisores}
\def\author{Lautaro Lasorsa}
\def\niveldia{}
\def\numero{}
\def\archivo{divisores}% Nombre del archivo
\def\version{}
%*********************************************************************
%*********************************************************************


\maketitle
\begin{multicols*}{2}

\section{Descripción del problema}

Gastón y Agustín están jugando un juego con divisores.

Al comenzar el juego, eligen un número $N$. 

Juegan por turnos, alternandose. En cada turno, si en ese momento están en el número $x$, el jugador que tiene el turno elige un entero positivo $y$ tal que $y$ divide a $x$ y $x$ dividido $y$ da un número primo.

Antes de iniciar el juego, además, Agustín y Gastón eligieron un número $M$, y a cada entero positivo $x$ le asignan un puntaje $ p_x = ((x^2)\%M)^2 $ (donde el  \% es la operación resto, es decir, a\%b es tomar el resto de la división entera entre a y b)

Llamemos $S$ a la suma de los puntajes asociados a los números por los que pasarón durante el juego (incluyendo el $N$ y el 1), el objetivo de Agustín es minimizar $S$, y el de Gastón es maximizarlo.

Ahora, lo que quieren es que desarrolles un programa que, dados $N$ y $M$, les indique cuanto valdrá $S$ en los casos donde Agustín o Gastón jueguen primero, asumiendo que ambos juegan siempre de forma optima segun sus objetivos.

\section{Detalles de implementación}
Debes implementar la función \texttt{Divisores(N,M)}, siendo $N$ y $M$ dos \texttt{ENTERO LARGO}. La función debe devolver un \texttt{VECTOR DE ENTERO LARGO} con 2 valores, el primero el valor de $S$ si Agustín juega el primer turno, y el segundo el valor de $S$ si Gastón juega el primer turno.

\section{Evaluador local}
El evaluador local leera primero 2 enteros, $N$ y $M$.

Luego llama a la función \texttt{Divisores(N,M)} y devuelve en una línea el contenido del vector devuelto por la función.

\section{Cotas}
  \begin{itemize}
     \item $ 1 \leq N \leq 10^{14}$
     \item $ 1 \leq M \leq 10^7$
  \end{itemize}

\section{Ejemplos}
Si el evaluador local recibe la siguiente entrada:

\caja{5.7cm}{\small
10 1
}

Una  implementación  correcta  devolvera:

\caja{5.7cm}{\small
0 0
}

En cambio, si recibe:

\caja{5.7cm}{\small
100 10
}


Podrá devolver:

\caja{5.7cm}{\small
51 26
}

Notar que a veces a los jugadores les conviene jugar segundos.
\section{Subtareas}

\begin{enumerate}
  \item $1 \leq N \leq 10$ (10 puntos)
  \item $1 \leq N \leq 1.000$ (15 puntos)
  \item $1 \leq N \leq 1.000.000$ (15 puntos)
  \item $1 \leq N \leq 10^{10}$ (20 puntos)
  \item Sin restricciones adicionales (40 puntos) 
\end{enumerate}

\end{multicols*}
\end{document}
